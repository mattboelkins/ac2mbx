\section{An Introduction to Differential Equations} \label{S:7.1.DEIntro}

\vspace*{-14 pt}
\framebox{\hspace*{3 pt}
\parbox{\boxwidth}{\begin{goals}
\item What is a differential equation and what
  kinds of information can it tell us?
\item How do differential equations arise in the world around us?
\item What do we mean by a solution to a differential equation?
\end{goals}} \hspace*{3 pt}}

\subsection*{Introduction}

In previous chapters, we have seen that a function's derivative tells
us the rate at which the function is changing.  More recently, 
the Fundamental Theorem of Calculus helped us to determine
the total change of a function over an interval 
when we know the function's rate
of change.  For instance, an object's velocity tells us the rate of
change of that object's position.  By integrating the velocity over a
time interval, we may determine by how much the position changes over
that time interval.  In particular, if we know where the object is at the beginning
of that interval, then we have enough information to accurately
predict where it will be at the end of the interval.

In this chapter, we will introduce the concept of \emph{differential equations} and explore
this idea in more depth.  Simply said, a differential equation is an equation that provides a
description of a function's derivative, which means that it tells us
the function's rate of change.  Using this information, we would like
to learn as much as possible about the function itself.  For instance,
we would ideally like to have an algebraic description of the
function.  As we'll see, this may be too much to ask in some
situations, but we will still be able to make accurate approximations.

\input{previews/7.1.PA1}

\subsection*{What is a differential equation?} \index{differential equation} 

A differential equation is an equation that describes the derivative,
or derivatives, of a function that is unknown to us.  
For instance, the equation
$$
\frac{dy}{dx} = x\sin x
$$
is a differential equation since it describes the derivative of a
function $y(x)$ that is unknown to us.

As many important examples of differential equations involve quantities
that change in time, the independent variable in our discussion will
frequently be time 
$t$.
For instance, in the preview activity, we considered the
differential equation
$$
\frac{ds}{dt} = 4t + 1.
$$
Knowing the velocity and the starting position of the object, we were
able to find the position at any later time.

Because differential equations describe the derivative of a function,
they give us information about how that function changes.  Our goal
will be to take this information and use it to predict the value of
the function in the future; in this way, differential equations
provide us with something like a crystal ball.

Differential equations arise frequently in our every day world.  For
instance, you may hear a bank advertising:

\begin{quote}{\em
    Your money will grow at a 3\% annual interest rate with us.
}
\end{quote}

This innocuous statement is really a differential equation.  Let's
translate:  $A(t)$ will be amount of money you have in your account at
time $t$.  On one hand, the rate at which your money grows is the
derivative $dA/dt$.  On the other hand, we are told that this rate is
$0.03 A$.  This leads to the differential equation
$$
\frac{dA}{dt} = 0.03 A.
$$

This differential equation has a slightly different feel than the
previous equation $\frac{ds}{dt} = 4t+1$.  In the
earlier example, the rate of change depends only on the independent
variable $t$, and we may find $s(t)$ by integrating the velocity $4t+1$.  
In the banking example, however, the rate of change depends
on the dependent variable $A$, so we'll need some new techniques in order to
find $A(t)$.  

\input{activities/7.1.Act1}

\subsection*{Differential equations in the world around us}
% index{ } %(if appropriate) 

As we have noted, differential equations give a natural way to
describe phenomena we see in the real world.  For instance, physical
principles are frequently expressed as a description of how a quantity
changes.  A good example is Newton's Second Law, an important physcial
principle that says:  

\begin{quote} {\em
    The product of an object's
    mass and acceleration equals the force applied to it.
}
\end{quote}

For instance, when gravity acts on an object near the earth's surface,
it exerts a force equal to $mg$, the mass of the object times the
gravitational constant $g$.  We therefore have
\begin{eqnarray*}
ma & = & mg, \ \mbox{or} \\
\quad \frac{dv}{dt} & = & g,
\end{eqnarray*}
where $v$ is the velocity of the object, and $g = 9.8$ meters per
second squared.  Notice that this physical principle does not tell us
what the object's 
velocity is, but rather how the object's velocity changes.  


\input{activities/7.1.Act2}

The point of this activity is to demonstrate how differential
equations model processes in the real world.  In this example, two
factors are influencing the velocities:  gravity and wind resistance.
The differential equation describes how these factors influence the
rate of change of the objects' velocities.

\subsection*{Solving a differential equation} \index{differential equation!solution}

We have said that a differential equation is an equation that
describes the derivative, 
or derivatives, of a function that is unknown to us.  By a {\em
  solution} to a differential equation, we mean simply a function that
satisies this description.

For instance, the first differential equation we looked at is
$$
\frac{ds}{dt} = 4t+1,
$$
which describes an unknown function $s(t)$.  We may check that $s(t) =
2t^2+t$ is a solution because it satisfies this description.  Notice
that $s(t) = 2t^2+t+4$ is also a solution.

If we have a candidate for a solution, it is straightforward to
check whether it is a solution or not.  Before we demonstrate,
however, let's consider the same issue in a simpler context.
Suppose we are given the equation
$2x^2 - 2x = 2x+6$ and asked whether $x=3$ is a solution.  To answer
this question, we could rewrite the variable $x$ in the equation with
the symbol $\Box$:
$$
2\Box^2 - 2\Box = 2\Box + 6.
$$
To determine whether $x=3$ is a solution, we can investigate the value of each side of the equation separately when the value $3$ is placed in $\Box$ and see if indeed the two resulting values are equal.  Doing so, we observe that
$$2\Box^2 - 2\Box = 2\cdot3^2 - 2\cdot3 = 12,$$
and
$$2\Box + 6 = 2\cdot3 + 6 = 12.$$
Therefore, $x=3$ is indeed a solution.

We will do the same thing with differential equations.  Consider
the differential equation 
\begin{eqnarray*}
\frac{dv}{dt} & = & 1.5 - 0.5v, \ \mbox{or} \\
\quad \frac{d\Box}{dt} & = & 1.5 - 0.5\Box.
\end{eqnarray*}
Let's ask whether $v(t) = 3 - 2e^{-0.5t}$ is a solution\footnote{At this time,
don't worry about why we chose this function;  we will learn
techniques for finding solutions to differential equations soon enough.  }. 
Using this formula for $v$, observe first that
$$\frac{dv}{dt} =  \frac{d\Box}{dt}  = \frac{d}{dt}[3 - 2e^{-0.5t}] = -2e^{-0.5t} \cdot (-0.5) = e^{-0.5t}$$
and
$$1.5 - 0.5v = 1.5 - 0.5\Box= 1.5 - 0.5(3 - 2e^{-0.5t}) = 1.5 - 1.5 + e^{-0.5t} = e^{-0.5t}.$$
Since $\frac{dv}{dt}$ and $1.5 - 0.5v$ agree for all values of $t$ when $v = 3-2e^{-0.5t}$, we have indeed
found a solution to the differential equation.

\input{activities/7.1.Act3}

This activity shows us something interesting.  Notice that the
differential equation has infinitely many solutions, which are
parametrized by the constant $C$ in $v(t) = 3+Ce^{-0.5t}$.  In Figure~\ref{F:7.1.family}, we see the graphs of these solutions for a few values of $C$, 
as labeled.

\begin{figure}[h]
\begin{center}
\includegraphics{figures/7_1_family.eps}
\caption{The family of solutions to the differential equation $\frac{dv}{dt} = 1.5 - 0.5v$.} \label{F:7.1.family} 
\end{center}
\end{figure}

Notice that the value of $C$ is connected to the initial value of the
velocity $v(0)$, since $v(0) = 3+C$.  In other words, while the
differential equation describes how the velocity changes as a function of the velocity itself, this is not enough information to determine the velocity
uniquely:  we also need to know the initial velocity.  For this
reason, differential equations will typically have infinitely many
solutions, one corresponding to each initial value.  We have seen this phenomenon before, such as when given the velocity of a moving object
$v(t)$, we were not able to uniquely determine the object's position
unless we also know its initial position.

If we are given a differential equation and an initial value for the
unknown function, we say that we have an {\em initial value problem.}
For instance,
$$
  \frac{dv}{dt} = 1.5-0.5v, \ v(0) = 0.5
$$
is an initial value problem.  In this situation, we know the value of
$v$ at one time and we know how $v$ is changing.  Consequently, there should
be exactly one function $v$ that satisfies the initial value problem.

This demonstrates the following important general property of initial value problems.

\vspace*{5pt}
\nin \framebox{\hspace*{3 pt}
\parbox{\boxwidth}{
    Initial value problems that are ``well behaved'' have exactly one
    solution, which exists in some interval around the initial point.
} \hspace*{3 pt}}
\vspace*{1pt}

We won't worry about what ``well behaved'' means---it is a technical
condition that will be satisfied by all the differential equations we
consider. 

To close this section, we note that differential equations may be
classified based on certain characteristics they may possess.  Indeed,
you may see many different types of differential equations in a
later course in differential equations.  For now, we would like to
introduce a few terms that are used to describe differential
equations.

A {\em first-order} differential equation\index{differential equation!first order} is one in which only the
first derivative of the function occurs.  For this reason,
$$
\frac{dv}{dt} = 1.5-0.5v
$$
is a first-order equation while
$$
\frac{d^2 y}{dt^2} = -10y
$$
is a second-order equation.

A differential equation is {\em autonomous}\index{autonomous} \index{differential equation!autonomous} if the independent
variable does not appear in the description of the derivative.  
For instance,
$$
\frac{dv}{dt} = 1.5-0.5v
$$
is autonomous because the description of the derivative $dv/dt$ does
not depend on time.   
The equation
$$
\frac{dy}{dt} = 1.5t - 0.5y,
$$
however, is not autonomous.

\newpage

\begin{summary}
\item A differential equation is simply an equation that describes the
  derivative(s) of an unknown function.
\item Physical principles, as well as some everyday situations, often
  describe how a quantity changes, which 
  lead to differential equations.
\item A solution to a differential equation is a function whose
  derivatives satisfy the equation's description.  Differential
  equations typically have infinitely many solutions, 
  parametrized by the initial values.
\end{summary}

\nin \hrulefill

\input{exercises/7.1.DEIntro(Ex)} 



\clearpage
