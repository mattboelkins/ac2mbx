\begin{activity} \label{A:6.5.1}  In this activity we explore the improper integrals $ \int_1^{\infty} \frac{1}{x} \, dx$ and $ \int_1^{\infty} \frac{1}{x^{3/2}} \, dx$.
\ba
	\item First we investigate $ \int_1^{\infty} \frac{1}{x} \, dx$.
	\be
		\item[i.] Use the First FTC to determine the exact values of $ \int_1^{10} \frac{1}{x} \, dx$, $ \int_1^{1000} \frac{1}{x} \, dx$, and $ \int_1^{100000} \frac{1}{x} \, dx$.  Then, use your calculator to compute a decimal approximation of each result.
		\item[ii.]  Use the First FTC to evaluate the definite integral $ \int_1^{b} \frac{1}{x} \, dx$ (which results in an expression that depends on $b$).
		\item[iii.]  Now, use your work from (ii.) to evaluate the limit given by
	$$\lim_{b \to \infty}  \int_1^{b} \frac{1}{x} \, dx.$$
	\ee
	\item Next, we investigate $ \int_1^{\infty} \frac{1}{x^{3/2}} \, dx$.
	\be
		\item[i.] Use the First FTC to determine the exact values of $ \int_1^{10} \frac{1}{x^{3/2}} \, dx$, $ \int_1^{1000} \frac{1}{x^{3/2}} \, dx$, and $ \int_1^{100000} \frac{1}{x^{3/2}} \, dx$.  Then, use your calculator to compute a decimal approximation of each result.
		\item[ii.]  Use the First FTC to evaluate the definite integral $ \int_1^{b} \frac{1}{x^{3/2}} \, dx$ (which results in an expression that depends on $b$).
		\item[iii.]  Now, use your work from (ii.) to evaluate the limit given by
	$$\lim_{b \to \infty}  \int_1^{b} \frac{1}{x^{3/2}} \, dx.$$
	\ee
	\item Plot the functions $y = \frac{1}{x}$ and $y = \frac{1}{x^{3/2}}$ on the same coordinate axes for the values $x = 0 \ldots 10$.  How would you compare their behavior as $x$ increases without bound?  What is similar?  What is different?
	\item How would you characterize the value of $ \int_1^{\infty} \frac{1}{x} \, dx$? of $ \int_1^{\infty} \frac{1}{x^{3/2}} \, dx$?  What does this tell us about the respective areas bounded by these two curves for $x \ge 1$?
\ea

\end{activity}
\begin{smallhint}
\ba
	\item Small hints for each of the prompts above.
\ea
\end{smallhint}
\begin{bighint}
\ba
	\item Big hints for each of the prompts above.
\ea
\end{bighint}
\begin{activitySolution}
\ba
	\item Solutions for each of the prompts above.
\ea
\end{activitySolution}
\aftera