\begin{activity} \label{A:7.4.2}  
  Suppose that a cup of coffee is initially at a temperature of
  105$^\circ$ F and is placed in a 75$^\circ$ F room.  Newton's law of
  cooling says that 
  $$
  \frac{dT}{dt} = -k(T-75),
  $$ 
  where $k$ is a constant of proportionality.

\ba
\item Suppose you measure that the coffee is cooling at one degree per
  minute at the time the coffee is brought into the room.  Use the
  differential equation to determine the value of the constant $k$.

\item Find all the solutions of this differential equation.

\item What happens to all the solutions as $t\to\infty$?  Explain how
  this agrees with your intuition.

\item What is the temperature of the cup of coffee after 20 minutes?

\item How long does it take for the coffee to cool to 80$^\circ$?

\ea
\end{activity}
\begin{smallhint}
\ba
	\item Small hints for each of the prompts above.
\ea
\end{smallhint}
\begin{bighint}
\ba
	\item Big hints for each of the prompts above.
\ea
\end{bighint}
\begin{activitySolution}
\ba
	\item Solutions for each of the prompts above.
\ea
\end{activitySolution}
\aftera