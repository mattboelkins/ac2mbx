\begin{activity} \label{8.3.Act1} Consider the series
\[\sum_{k=1}^{\infty} \frac{1}{k^2}.\]
While it is physically impossible to add an infinite collection of numbers, we can, of course, add any finite collection of them.  In what follows, we investigate how understanding how to find the $n$th partial sum (that is, the sum of the first $n$ terms) enables us to make sense of the infinite sum.
\ba
\item Sum the first two numbers in this series. That is, find a numeric value for
\[\sum_{k=1}^2 \frac{1}{k^2}\]

\item Next, add the first three numbers in the series.

\item Continue adding terms in this series to complete Table~\ref{T:8.3.1_part_sum_ex}. Carry each sum to at least 8 decimal places.
\begin{table}[ht]
\begin{center}
\renewcommand{\arraystretch}{1.5}
\begin{tabular}{r c p{0.5in} p{1.0in} r c p{0.5in}}
$\ds \sum_{k=1}^{1} \frac{1}{k^2}$   & = & $1$  &   &$\ds \sum_{k=1}^{6} \frac{1}{k^2}$   &= & \\
$\ds \sum_{k=1}^{2} \frac{1}{k^2}$   & = &     &   &$\ds \sum_{k=1}^{7} \frac{1}{k^2}$   & = & \\
$\ds \sum_{k=1}^{3} \frac{1}{k^2}$   & = &     &   &$\ds \sum_{k=1}^{8} \frac{1}{k^2}$   & = &  \\
$\ds \sum_{k=1}^{4} \frac{1}{k^2}$   & = &     &   &$\ds \sum_{k=1}^{9} \frac{1}{k^2}$   &  = &\\
$\ds \sum_{k=1}^{5} \frac{1}{k^2}$   & = &     &   &$\ds \sum_{k=1}^{10} \frac{1}{k^2}$  & = & \\
\end{tabular}
\caption{Sums of some of the first terms of the series $\sum_{k=1}^{\infty} \frac{1}{k^2}$} \label{T:8.3.1_part_sum_ex}
\end{center}
\end{table}

\item The sums in the table in (c) form a sequence whose $n$th term is $S_n =  \sum_{k=1}^{n} \frac{1}{k^2}$. Based on your calculations in the table, do you think the sequence $\{S_n\}$ converges or diverges? Explain. How do you think this sequence $\{S_n\}$ is related to the series $ \sum_{k=1}^{\infty} \frac{1}{k^2}$?

\ea
\end{activity}

\begin{smallhint}
\ba
	\item Small hints for each of the prompts above.
\ea
\end{smallhint}
\begin{bighint}
\ba
	\item Big hints for each of the prompts above.
\ea
\end{bighint}
\begin{activitySolution}
\ba
	\item See the Table in part (c).
    \item See the Table in part (c).
    \item If we add the first few terms of the sequence $\left\{\frac{1}{k^2}\right\}$ we obtain the entries (to 10 decimal places) in the following table.
%\begin{table}[ht]
\begin{center}
\renewcommand{\arraystretch}{1.5}
\begin{tabular}{c|c p{1.0in} c|c}
$\ds \sum_{k=1}^{1} \frac{1}{k^2}$   & $1$              &   &$\ds \sum_{k=1}^{6} \frac{1}{k^2}$   & $1.491388889$ \\
$\ds \sum_{k=1}^{2} \frac{1}{k^2}$   & $1.25$           &   &$\ds \sum_{k=1}^{7} \frac{1}{k^2}$   & $1.511797052$ \\
$\ds \sum_{k=1}^{3} \frac{1}{k^2}$   & $1.361111111$    &   &$\ds \sum_{k=1}^{8} \frac{1}{k^2}$   & $1.527422052$ \\
$\ds \sum_{k=1}^{4} \frac{1}{k^2}$   & $1.423611111$    &   &$\ds \sum_{k=1}^{9} \frac{1}{k^2}$   & $1.539767731$ \\
$\ds \sum_{k=1}^{5} \frac{1}{k^2}$   & $1.463611111$    &   &$\ds \sum_{k=1}^{10} \frac{1}{k^2}$  & $1.549767731$ \\
\end{tabular}
%\label{T:8.3.1_part_sum_ex_b}
%\caption{Sums of some of the first terms of the series $\sum_{k=1}^{\infty} \frac{1}{k^2}$}
\end{center}
%\end{table}
    \item These sums in the table in part (c) seem to indicate that the sequence $\{S_n\}$ converges to something a bit larger than 1.5. Since the sequence $\{S_n\}$ is found by adding up the first $k$ terms of the series, we should expect that the series $\ds \sum_{k=1}^{n} \frac{1}{k^2}$ is the limit of the sequence $\{S_n\}$ as $n$ goes to infinity.
\ea
\end{activitySolution}
\aftera 