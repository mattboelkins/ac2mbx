\begin{activity} \label{A:7.4.skydiver}  
      \item A dose of morphine is 
        absorbed from the bloodstream of a patient at a rate
        proportional to the amount in the bloodstream.  

        \ba
        \item Write a differential equation for $M(t)$, the amount of
          morphine in the patient's bloodstream, using $k$ as the
          constant proportionality.
        \item 
          Assuming that the initial dose of morphine is $M_0$,
          solve the initial value problem to find $M(t)$.  Use the
          fact that the half-life for the absorption of morphine is
          two hours to find the constant $k$.
        \item Suppose that a patient is given morphine intraveneously
          at the rate of 3 milligrams per hour.  Write a differential
          equation that combines the intraveneous administration of
          morphine with the body's natural absorption.
        \item Find any equilibrium solutions and determine their
          stability. 
        \item Assuming that there is initially no morphine in the
          patient's bloodstream, solve the initial value problem to
          determine $M(t)$.
        \item What happens to $M(t)$ after a very long time?
        \item Suppose that a doctor asks you to reduce the
          intraveneous rate so that there is eventually 7 milligrams
          of morphine in the patient's body.  To what rate would you reduce
          the intraveneous flow?
        \ea





\ea
\end{activity}
\begin{smallhint}
\ba
	\item Small hints for each of the prompts above.
\ea
\end{smallhint}
\begin{bighint}
\ba
	\item Big hints for each of the prompts above.
\ea
\end{bighint}
\begin{activitySolution}
\ba
	\item Solutions for each of the prompts above.
\ea
\end{activitySolution}
\aftera