\begin{activity} \label{A:7.4.salt}  
  Suppose that you have a water tank that holds 100 gallons of water.
  A briny solution, which contains 20 grams of salt per gallon, enters
  the tank at the rate of 3 gallons per minute.

  At the same time, the solution is well mixed, and water is pumped
  out of the tank at the rate of 3 gallons per minute.

\ba
\item Since 3 gallons enters the tank every minute and 3 gallons
  leaves every minute, what can you conclude about the volume of water
  in the tank.

\item How many grams of salt enters the tank every minute?

\item Suppose that $S(t)$ denotes the number of grams of salt in the
  tank in minute $t$.  How many grams are there in each gallon in
  minute $t$?

\item Since water leaves the tank at 3 gallons per minute, how many
  grams of salt leave the tank each minute? 

\item Write a differential equation that expresses the total rate of
  change of $S$.

\item Identify any equilibrium solutions and determine whether they
  are stable or unstable.

\item Suppose that there is initially no salt in the tank.  Find the
  amount of salt $S(t)$ in minute $t$.

\item What happens to $S(t)$ after a very long time?  Explain how you
  could have predicted this only knowing how much salt there is in
  each gallon of the
  briny solution that enters the tank.

\ea
\end{activity}
\begin{smallhint}
\ba
	\item Small hints for each of the prompts above.
\ea
\end{smallhint}
\begin{bighint}
\ba
	\item Big hints for each of the prompts above.
\ea
\end{bighint}
\begin{activitySolution}
\ba
	\item Solutions for each of the prompts above.
\ea
\end{activitySolution}
\aftera