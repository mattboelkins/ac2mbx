\section{Probability} \label{S:6.6.Prob}

\vspace*{-14 pt}
\framebox{\hspace*{3 pt}
\parbox{6.25 in}{\begin{goals}
  \item What is a probability density function?
  \item What is a cumulative distribution function?
  \item How does the subject of probability tie together such key calculus ideas as density, area under a curve, improper integrals, and the Second FTC?
\end{goals}} \hspace*{3 pt}}

\subsection*{Introduction}

Reading SAT data.

\input{previews/6.6.PA1}

\subsection*{More on density functions} \index{density function}

In Section~\ref{S:6.3.Mass}, we learned that if $\rho(x)$ measures the mass density distribution of a quantity along an axis, say with units ``grams per centimeter,'' then $\int_a^b \rho(x) \, dx$ measures the total mass of the quantity that lies between $a$ and $b$.  Certainly it makes sense to think about how other quantities might be distributed relative to a given variable.  For instance, we might be interested in how people are distributed relative to their height or age.  

Mean female height: 64 in.
Standard deviation: 2.8


\input{activities/6.6.Act1}

\subsection*{Probability Density Functions} \index{probability density function}


\nin In the two examples we've considered so far, the functions we have considered are called \emph{probability density functions}.

\be
  \item A function $p(x)$ is a {\bf probability density function} provided that
	\begin{itemize}
	  \item \ \\	
 	  \item \ \\
	  \item \ \\ \ \\
	\end{itemize}
 The \emph{mean} of a characteristic of a population given by a density function $p(x)$ is


(Note well how this is related to the center of mass of a physical object whose density we know.)
 The \emph{median} of a characteristic of a population given by a density function $p(x)$ is
  


%\input{activities/6.6.Act2}

\subsection*{Cumulative Distribution Functions} \index{cumulative distribution function}

%\input{activities/6.6.Act3}


%\nin \framebox{\hspace*{3 pt}
%\parbox{6.25 in}{
\begin{summary}
  \item 
\end{summary}
%} \hspace*{3 pt}}

\nin \hrulefill

\input{exercises/6.6.Prob(Ex)} 

\clearpage
