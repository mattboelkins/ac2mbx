\section{Probability} \label{S:6.6.Prob}

\vspace*{-14 pt}
\framebox{\hspace*{3 pt}
\parbox{6.25 in}{\begin{goals}
  \item What is a probability density function?
  \item What is a cumulative distribution function?
  \item How does the subject of probability tie together such key calculus ideas as density, area under a curve, improper integrals, and the Second FTC?
\end{goals}} \hspace*{3 pt}}

\subsection*{Introduction}

Reading SAT data.

\begin{pa} \label{PA:6.6}  SAT data problem
\ba
	\item Determine the fraction of students who 
\ea
\end{pa} 
\afterpa

\subsection*{More on density functions} \index{density function}

In Section~\ref{S:6.3.Mass}, we learned that if $\rho(x)$ measures the mass density distribution of a quantity along an axis, say with units ``grams per centimeter,'' then $\int_a^b \rho(x) \, dx$ measures the total mass of the quantity that lies between $a$ and $b$.  Certainly it makes sense to think about how other quantities might be distributed relative to a given variable.  For instance, we might be interested in how people are distributed relative to their height or age.  

Mean female height: 64 in.
Standard deviation: 2.8


\begin{activity} \label{A:6.6.1}  Suppose that for a certain population in a given country, the age density function $p(x)$ is given by the function $$p(x) = 0.002911 \left( \frac{1}{2048}x^4-\frac{1}{64}x^2+10 \right)e^{-0.1313x}, \ 0 \le x \le 100.$$
The units on $p$ are ``fraction of the population per year of age'' where $x$ measures the age of a person in years.  A graph of $p$ is shown in Figure~\ref{F:}
\ba
	\item What fraction of the population is between the ages of 10 and 12?  Write a definite integral and use computational technology to evaluate it.
	  \item Make two observations about trends in the population based on the graph of this density function.
	  \item Use computational technology to evaluate the integral expression
	  $$\frac{\int_0^{105} xp(x) \, dx}{\int_0^{105} p(x) \, dx}.$$
	  What is the meaning of the value you found?
	  \item Use computational technology to estimate the value of $M$ for which $\int_0^M p(x) \, dx = \frac{1}{2}$.  What is the meaning of the value $M$ in the context of the problem?
\ea

\end{activity}
\begin{smallhint}
\ba
	\item Small hints for each of the prompts above.
\ea
\end{smallhint}
\begin{bighint}
\ba
	\item Big hints for each of the prompts above.
\ea
\end{bighint}
\begin{activitySolution}
\ba
	\item Solutions for each of the prompts above.
\ea
\end{activitySolution}
\aftera

\subsection*{Probability Density Functions} \index{probability density function}


\nin In the two examples we've considered so far, the functions we have considered are called \emph{probability density functions}.

\be
  \item A function $p(x)$ is a {\bf probability density function} provided that
	\begin{itemize}
	  \item \ \\	
 	  \item \ \\
	  \item \ \\ \ \\
	\end{itemize}
 The \emph{mean} of a characteristic of a population given by a density function $p(x)$ is


(Note well how this is related to the center of mass of a physical object whose density we know.)
 The \emph{median} of a characteristic of a population given by a density function $p(x)$ is
  


%\input{activities/6.6.Act2}

\subsection*{Cumulative Distribution Functions} \index{cumulative distribution function}

%\input{activities/6.6.Act3}


%\nin \framebox{\hspace*{3 pt}
%\parbox{6.25 in}{
\begin{summary}
  \item 
\end{summary}
%} \hspace*{3 pt}}

\nin \hrulefill

\begin{exercises} 
  \item The probability of a transistor failing between months $t = a$ and $t = b$ is given by the probability density function $p(t) = ce^{-ct}$, where this formula is valid for all $t \ge 0$ and $p(t) = 0$ for all $t < 0$.
\ba
	\item If the probability of the transistor failing in the first 6 months is 0.1, what is the value of $c$?
	\item Suppose that for a different transistor, the value of $c$ is known to be $c = 0.05$.   For this $c$ value, find the mean time it takes for a transistor to fail.
	\item For the same $c$-value ($c = 0.05$), find the median time it takes for a transistor to fail. 
\ea

\item  A probability density function is given by $p(x) = \frac{1}{4a^2} (x-a)^2$, where $a > 0$ and the formula is valid of $0 \le x \le a$, while $p(x) = 0$ if $x < 0$ or $x > a$.  Suppose that $p(x)$ models the amount of time, in minutes, a customer waits at a drive-in bank teller.
\ba
	\item What is the value of $a$?
	\item For the (similar, but different) pdf $p(x) = \frac{3}{125} (x-5)^2$ (which is valid for $0 \le x \le 5$, otherwise $p(x) = 0$), find the mean wait time.
	\item Determine the median wait time.
	\item Find a formula that does not involve an integral for the cdf, $P$, that corresponds to this pdf.
\ea	
  
\end{exercises}
\afterexercises
 

\clearpage
