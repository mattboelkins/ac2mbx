\section{Population Growth and the Logistic Equation} 
\label{S:7.6.Logistic}

\vspace*{-14 pt}
\framebox{\hspace*{3 pt}
\parbox{\boxwidth}{\begin{goals}
\item How can we use differential equations to realistically model the growth of a
  population? 
\item How can we assess the accuracy of our models?
\end{goals}} \hspace*{3 pt}}

\subsection*{Introduction}

The growth of the earth's population is one of the pressing issues of
our time.  Will the population continue to grow?  Or will it perhaps 
level off at some point, and if so, when?  In this section, we
will look at two ways in which we may use differential equations to
help us address questions such as these.

Before we begin, let's consider again two important differential
equations that we have seen in earlier work this chapter.

\input{previews/7.6}

\subsection*{The earth's population}

We will now begin studying the earth's population.  To get started,
here are some data for the earth's population in recent years that we
will use in our investigations.

\begin{center}
  \begin{tabular}{|c|c|}
    \hline
    Year & Population (in billions) \\
    \hline
    1998 & 5.932 \\
    1999 & 6.008 \\
    2000 & 6.084 \\
    2001 & 6.159 \\
    2002 & 6.234 \\
    2005 & 6.456 \\
    2006 & 6.531 \\
    2007 & 6.606 \\
    2008 & 6.681 \\
    2009 & 6.756 \\
    2010 & 6.831 \\
    \hline
  \end{tabular}
\end{center}

\input{activities/7.6.exponential}

Our work in Activity~\ref{A:7.6.1} shows that that the exponential model is fairly accurate for years 
relatively close to 2000.  However, if we go too far into the
future, the model predicts increasingly large rates of change, which
causes the population to grow arbitrarily large.  This does not make
much sense since it is unrealistic to expect that the earth would be able to support
such a large population.  

The constant $k$ in the differential equation has an important
interpretation.  Let's rewrite the differential equation $\frac{dP}{dt} = kP$ by solving for $k$, so that we have
$$k = \frac{dP/dt}{P}.$$
Viewed in this light, $k$ is the ratio of the rate of change to the
population;  in other words, it is the contribution to the rate of change 
from a single person.  We call this the {\em per capita
  growth rate}\index{per capita growth rate}.

In the exponential model we introduced in Activity~\ref{A:7.6.1}, the per capita growth rate is
constant.  In particular, we are assuming that when the population is
large, the per capita growth rate is the same as when the population
is small.  It is natural to think that the per capita growth rate should
decrease when the population becomes large, since there will not be
enough resources to support so many people.  In other words, we expect that a more realistic model would hold if we assume
that the per capita growth rate depends on the population $P$.

In the previous activity, we computed the per capita growth rate in a
single year by computing $k$, the quotient of $\frac{dP}{dt}$ and $P$ (which we did for $t = 0$).  If we return data and compute the per capita
growth rate over a range of years, we generate the data shown in Figure~\ref{F:7.6.census}, which shows how the per capita growth rate is a function of the population, $P$.  
\begin{figure}[h]
\begin{center}
  \includegraphics{figures/7_6_census.eps}
\end{center}
\caption{A plot of per capita growth rate vs.~population $P$.} \label{F:7.6.census}
\end{figure}
From the data, we see that the per capita growth rate appears to decrease as
the population increases.  In fact, the points seem to lie very close
to a line, which is shown at two different scales in Figure~\ref{F:7.6.census1}.
\begin{figure}[h]
\begin{center}
  \includegraphics{figures/7_6_census_1.eps} \qquad
  \includegraphics{figures/7_6_census_2.eps} \qquad
\end{center}
\caption{The line that approximates per capita growth as a function of population, $P$.} \label{F:7.6.census1}
\end{figure}
Looking at this line carefully, we can find its equation to be
$$
\frac{dP/dt}{P} = 0.025 - 0.002P.
$$
If we multiply both sides by $P$, we arrive at the differential
equation
$$
\frac{dP}{dt} = P(0.025 - 0.002P).
$$
Graphing the dependence of $dP/dt$ on the population $P$, we see that this differential equation demonstrates a quadratic relationship between $\frac{dP}{dt}$ and $P$, as shown in Figure~\ref{F:7.6.logistic}.
\begin{figure}[h]
\begin{center}
  \includegraphics{figures/7_6_logistic_de.eps}
\end{center}
\caption{A plot of $\frac{dP}{dt}$ vs.~$P$ for the differential equation $\frac{dP}{dt} = P(0.025 - 0.002P)$.}\label{F:7.6.logistic}
\end{figure}
The equation $\frac{dP}{dt} = P(0.025 - 0.002P)$ is an example of the {\em logistic equation}, 
and is the second model for population growth that we will consider.  We
have reason to believe that it will be more realistic since the per
capita growth rate is a decreasing function of the population.

Indeed, the graph in Figure~\ref{F:7.6.logistic} shows that there are two equilibrium
solutions, $P=0$, which is unstable, and $P=12.5$, which is a stable
equilibrium.  The graph shows that any solution with $P(0) >0$ will
eventually stabilize around 12.5.  In other words, our model predicts
the world's population will eventually stabilize around 12.5
billion.

A prediction for the long-term behavior of the population is a
valuable conclusion to draw from our differential equation.  We would,
however, like to answer some quantitative questions.  For instance,
how long will it take to reach a population of 10 billion?  To determine this,
we need to find an explicit solution of the equation.  

\subsection*{Solving the logistic differential equation} \index{logistic}

Since we would like to apply the logistic model in more general situations, we state the logistic equation\index{logistic equation} in its more general form,
\begin{equation} \label{E:7.6.logistic}
\frac{dP}{dt} = kP(N-P).
\end{equation}
The equilibrium solutions here are when $P=0$ and $1-\frac PN = 0$,
which shows that $P=N$.  The equilibrium at $P=N$ is called the {\em
  carrying capacity}\index{carrying capacity} of the population for it represents the stable
population that can be sustained by the environment.

We now solve the logistic equation~(\ref{E:7.6.logistic}).  The equation is separable, so we separate the variables
$$
\frac{1}{P(N-P)}\frac{dP}{dt} = k,
$$
and integrate to find that
$$
\int \frac{1}{P(N-P)}~dP = \int k~dt.
$$

To find the antiderivative on the left, we use the partial fraction decomposition
$$
\frac{1}{P(N-P)} = \frac 1N\left[\frac 1P + \frac 1{N-P}\right].
$$
Now we are ready to integrate, with 
$$
\int \frac 1N\left[\frac 1P + \frac 1{N-P}\right] ~dP  =  \int k~dt. 
$$
On the left, observe that $N$ is constant, so we can remove the factor of $\frac{1}{N}$ and antidifferentiate to find that
$$
\frac 1N (\ln|P| - \ln|N-P|)  =  kt + C.
$$
Multiplying both sides of this last equation by $N$ and using an important rule of logarithms, we next find that
$$\ln\left|\frac{P}{N-P}\right| = kNt + C.$$
From the definition of the logarithm, replacing $e^C$ with $C$, and letting $C$ absorb the absolute value signs, we now know that 
$$\frac{P}{N-P} =  Ce^{kNt}.$$
At this point, all that remains is to determine $C$ and solve algebraically for $P$.

If the initial population is $P(0) = P_0$, then it follows that $C = \frac{P_0}{N-P_0}$, so
$$\frac{P}{N-P} = \frac{P_0}{N-P_0}e^{kNt}.$$
We will solve this most recent equation for $P$ by multiplying both sides by
$(N-P)(N-P_0)$ to obtain 
\begin{eqnarray*}
P(N-P_0) &=& P_0(N-P)e^{kNt}  \\
	 &=& P_0Ne^{kNt} - P_0Pe^{kNt}. 
\end{eqnarray*}	 
Swapping the left and right sides, expanding, and factoring, it follows that
\begin{eqnarray*}
P_0Ne^{kNt} & = & P(N-P_0) + P_0 Pe^{kNt}  \\
	& = & P(N-P_0 + P_0e^{kNt}). 
\end{eqnarray*}
Dividing to solve for $P$, we see that
$$P = \frac{P_0Ne^{kNt}}{N-P_0 + P_0e^{kNt}}.$$
Finally, we choose to multiply the numerator and denominator by $\frac{1}{P_0}e^{-kNt}$
to obtain
$$
P(t) = \frac{N}{\left(\frac{N-P_0}{P_0}\right) e^{-kNt} + 1}.
$$

While that was a lot of algebra, notice the result:  we have
found an explicit solution to the initial value problem
$$
  \frac{dP}{dt} = kP(N-P), \ P(0) = P_0,
$$
and that solution\index{logistic equation!solution} is 
\begin{equation}\label{E:7.6.logistic_solution}
P(t) = \frac{N}{\left(\frac{N-P_0}{P_0}\right) e^{-kNt} + 1}.
\end{equation}

For the logistic equation describing the earth's population that we worked with earlier in this section, we have
$$k=0.002, \quad N= 12.5, \quad \hbox{and} \quad P_0 = 6.084.$$
This gives the solution
$$
P(t) = \frac{12.5}{1.0546e^{-0.025t} + 1},
$$
whose graph is shown in Figure~\ref{F:7.6.logistic_sol}
\begin{figure}[h]
\begin{center}
  \includegraphics{figures/7_6_logistic_sol.eps}
\end{center}
\caption{The solution to the logistic equation modeling the earth's population.} \label{F:7.6.logistic_sol}
\end{figure}
Notice that the graph shows the population leveling off at 12.5 billion, as
we expected, and that the population will be around 10 billion in the
year 2050.  These results, which we have found using a relatively simple
mathematical model, agree fairly well with predictions made using a
much more sophisticated model developed by the United Nations.

The logistic equation is useful in other situations, too, as it is good for modeling any situation in which limited growth is possible.  For instance, it could model the spread of a flu virus through a population contained on a cruise ship, the rate at which a rumor spreads within a small town, or the behavior of an animal population on an island.  Again, it is important to realize that through our work in this section, we have completely solved the logistic equation, regardless of the values of the constants $N$, $k$, and $P_0$.  Anytime we encounter a logistic equation, we can apply the formula we found in Equation~(\ref{E:7.6.logistic_solution}).

\input{activities/7.6.logistic}

\begin{summary}
\item If we assume that the rate of growth of a population is
  proportional to the population, we are led to a model in which the
  population grows without bound and at a  rate that grows without bound. 
\item By assuming that the per capita growth rate decreases as the
  population grows, we are led to the logistic model of population
  growth, which predicts that the population will eventually
  stabilize at the carrying capacity.
\end{summary}

\nin \hrulefill

\input{exercises/7.6} 




\clearpage
