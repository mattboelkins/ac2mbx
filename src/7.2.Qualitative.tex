\section{Qualitative behavior of solutions to DEs} \label{S:7.2.Qualitative}

\vspace*{-14 pt}
\framebox{\hspace*{3 pt}
\parbox{\boxwidth}{\begin{goals}
\item What is a slope field?  
\item How can we use a slope field to obtain qualitative information
  about the solutions of a differential equation?
\item What are stable and unstable equilibrium solutions of an
  autonomous differential equation? 
\end{goals}} \hspace*{3 pt}}

\subsection*{Introduction}

In earlier work, we have used the tangent line to the graph of a
function $f$ at a point $a$ to approximate the values of $f$ near $a$.
The usefulness of this approximation is that we need to know very
little about the function; armed with only the value $f(a)$ and the
derivative $f'(a)$, we may find the equation of the tangent line and
the approximation  
$$
f(x) \approx f(a) + f'(a)(x-a).
$$ 

Remember that a first-order differential equation gives us information
about the derivative of an unknown function. 
Since the derivative at a point tells us the slope of the
tangent line at this point, a differential equation
gives us crucial information about the tangent lines to the graph of
a solution.  We will use this information about the tangent lines to
create a \emph{slope field} for the differential equation, which enables
us to sketch solutions to initial value problems.  Our aim will be to
understand the solutions qualitatively.  That is, we would like to
understand the basic nature of solutions, such as their long-range
behavior, without precisely determining the value of a solution at a
particular point.

\input{previews/7.2.PA1}

\subsection*{Slope fields} \index{slope field}

Preview Activity~\ref{PA:7.2} shows that we may sketch the solution to an initial
value problem if we know an appropriate collection of tangent lines.  Because we may use a given differential equation to determine the slope of the tangent
line at any point of interest, by plotting a useful collection of these, we can get an accurate sense of how certain solution curves must behave.

Let's continue looking at the differential equation $
\ds \frac{dy}{dt} = t-2.
$
If $t=0$, this equation says that $dy/dt = 0-2=-2$.  Note that this value holds regardless of the value of $y$.  We will therefore
sketch tangent lines for several values of $y$ and $t=0$ with a slope
of $-2$. 

\begin{center}
  \includegraphics{figures/7_2_field_0.eps}
\end{center}

Let's continue in the same way:  if $t=1$, the differential equation
tells us that $dy/dt = 1-2=-1$, and this holds regardless of the value of $y$.  We now sketch tangent lines for
several values of $y$ and $t=1$ with a slope of $-1$.

\begin{center}
  \includegraphics{figures/7_2_field_1.eps}
\end{center}

Similarly, we see that when $t=2$, $dy/dt = 0$ and when $t=3$,
$dy/dt=1$.  We may therefore add to our growing collection of tangent line plots to achieve the next figure.

\begin{center}
  \includegraphics{figures/7_2_field_3.eps}
\end{center}

In this figure, you may see the solutions to the differential
equation emerge.  However, for the sake of clarity, we will
add more tangent lines to provide the more complete picture shown below.

\begin{center}
  \includegraphics{figures/7_2_field_23a.eps}
\end{center}

This most recent figure, which is called a {\em slope field}\index{slope field} for the differential
equation, allows us to sketch
solutions just as we did in the preview activity.  Here, we will begin with the
initial value $y(0) = 1$ and start sketching the solution by following
the tangent line, as shown in the next figure.

\begin{center}
  \includegraphics{figures/7_2_field_30.eps}
\end{center}

We then continue using this principle:  whenever the solution passes
through a point at which a tangent line is drawn, that line is tangent
to the solution.  Doing so leads us to the following sequence of images.

\begin{center}
    \includegraphics{figures/7_2_field_31.eps} \hspace{0.25in}
    \includegraphics{figures/7_2_field_32.eps} \hspace{0.25in}
    \includegraphics{figures/7_2_field_33.eps}
\end{center}

In fact, we may draw solutions for any possible initial value, and doing this for several different initial values for $y(0)$ results in the graphs shown next.
    
\begin{center}
  \includegraphics{figures/7_2_field_4.eps}
\end{center}

Just as we have done for the most recent example with $\frac{dy}{dt} = t-2$, we can construct a slope field for any differential equation of interest.  The slope field provides us with visual information about how we expect solutions to the differential equation to behave.

\input{activities/7.2.Act1}

\subsection*{Equilibrium solutions and stability} 

As our work in Activity~\ref{A:7.2.1} demonstrates, first-order autonomous
solutions may have solutions that are constant.  In fact, these are
quite easy to detect by inspecting the differential equation $dy/dt =
f(y)$:  constant solutions necessarily have a zero derivative so 
$dy/dt = 0 = f(y)$.  

For example, in Activity~\ref{A:7.2.1}, we considered the
equation
$$
\frac{dy}{dt} = f(y)=-\frac12(y-4).
$$
Constant solutions are found by setting $f(y) = -\frac12(y-4) = 0$,
which we immediately see implies that $y = 4$.  

Values of $y$ for which $f(y) = 0$ in an autonomous differential equation $\frac{dy}{dt} = f(y)$ are usually called or {\em equilibrium solutions} \index{equilibrium solution} of the differential
equation.  

\input{activities/7.2.Act2}

\begin{summary}
\item A slope field is a plot created by graphing the tangent lines of
  many different solutions to a differential equation.
\item Once we have a slope field, we may sketch the graph of solutions
  by drawing a curve that is always tangent to the lines in the slope
  field. 
\item Autonomous differential equations sometimes have constant
  solutions that we call 
  equilibrium solutions.  These may be classified as stable or
  unstable, depending on the behavior of nearby solutions.
\end{summary}

\nin \hrulefill

\input{exercises/7.2} 



\clearpage
