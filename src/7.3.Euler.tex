\section{Euler's method} \label{S:7.3.Euler}

\vspace*{-14 pt}
\framebox{\hspace*{3 pt}
\parbox{\boxwidth}{\begin{goals}
\item What is Euler's method and how can we use it to approximate the
  solution to an initial value problem?  
\item How accurate is Euler's method?
\end{goals}} \hspace*{3 pt}}

\subsection*{Introduction}

In Section~\ref{S:7.2.Qualitative}, we saw how a slope field can be used to sketch
solutions to a differential equation.  In particular, the slope field
is a plot of a large collection of tangent lines to a large number of solutions of the
differential equation, and we sketch a single solution by simply following
these tangent lines.  With a little more thought, we may use this same
idea to numerically approximate the solutions of a differential equation.  

\input{previews/7.3.PA1}

\subsection*{Euler's Method} \index{Euler's Method}

Preview Activity~\ref{PA:7.3} demonstrates the essence of an algorithm, which is known as
Euler's Method, that generates a numerical approximation to the solution
of an initial value problem.\footnote{``Euler'' is pronounced
  ``Oy-ler.''  Among other things, Euler is the mathematician credited with the famous number $e$; if you incorrectly pronounce his name ``You-ler,'' you fail to appreciate his genius and legacy.}  In this algorithm, we
will approximate the solution by taking horizontal steps of a fixed
size that we denote by $\Delta t$.  

Before explaining the algorithm in detail, let's remember how we
compute the slope of a line:  the slope is the ratio of the vertical
change to the horizontal change, as shown in the following figure.

\begin{center}
  \includegraphics{figures/7_3_slope.eps}
\end{center}
In other words, $m = \frac{\Delta y}{\Delta t}$.  Said differently, the
vertical change is the product of the slope and the horizontal change:
$$\Delta y = m\Delta t.$$

Suppose that we would like to solve the initial value problem
$$
  \frac{dy}{dt} = t - y, \ y(0) = 1.
$$
While there is an algorithm by which we can find an algebraic formula for the solution to this initial value problem, and we can check that this solution is $y(t) = t -1 + 2e^{-t}$, we are instead interested in generating an approximate solution by creating a sequence
of points $(t_i, y_i)$, where $y_i\approx y(t_i)$.  For this first example, we choose $\Delta t = 0.2$.

\medskip

\noindent \begin{minipage}[l]{3.6in}
Since we know that $y(0) = 1$, we will take the initial point to be
$(t_0,y_0) = (0,1)$ and move horizontally by $\Delta t = 0.2$ to the
point $(t_1,y_1)$.  Therefore, $t_1=t_0+\Delta t = 0.2$. 
The differential equation tells us that the slope of the 
tangent line at this point is
$$
m=\frac{dy}{dt}\bigg\vert_{(0,1)} = 0-1 = -1.
$$
Therefore, if we move along the tangent line by taking a horizontal
step of size $\Delta t=0.2$, we must also move vertically by
$$\Delta y = m\Delta t = -1\cdot 0.2 = -0.2.$$  We then have the
approximation $y(0.1) \approx y_1= y_0 + \Delta y = 1 - 0.2 = 0.8.$  At this point, we have executed one step of Euler's method.
\end{minipage}
\hspace*{.25in}
\begin{minipage}[l]{2in}
\includegraphics{figures/7_3_euler_points_1.eps}
\end{minipage}

\medskip
\noindent \begin{minipage}[l]{3.6in}
Now we repeat this process:  at $(t_1,y_1) = (0.2,0.8)$, the
differential equation tells us that the slope is 
$$
m=\frac{dy}{dt}\bigg\vert_{(0.2,0.8)} = 0.2-0.8 = -0.6.
$$
If we move horizontally by $\Delta t$ to $t_2=t_1+\Delta = 0.4$, we must move
vertically by
$$
\Delta y = -0.6\cdot0.2 = -0.12.
$$
We consequently arrive at $y_2=y_1+\Delta y = 0.8-0.12 = 0.68$, which
gives $y(0.2)\approx 0.68$.  Now we have completed the second step of Euler's method.


\end{minipage}
\hspace*{0.25in}
\begin{minipage}[c]{2in}
\includegraphics{figures/7_3_euler_points_2.eps}
\end{minipage}

\medskip

If we continue in this way, we may generate the points $(t_i, y_i)$
shown at left in Figure~\ref{F:7.3.EulerPts}.  In situations where we are able to find a formula for the actual solution $y(t)$, we can graph $y(t)$ to compare it to the points generated by Euler's method, as shown at right in Figure~\ref{F:7.3.EulerPts}.
\begin{figure}[h]
\begin{center}
\scalebox{0.9}{\includegraphics{figures/7_3_euler_points_6.eps}}  \hspace{0.25in}
  \scalebox{0.9}{\includegraphics{figures/7_3_euler.eps}}
\end{center}
\caption{At left, the points and piecewise linear approximate solution generated by Euler's method; at right, the approximate solution compared to the exact solution (shown in blue).} \label{F:7.3.EulerPts}
\end{figure}


Because we need to generate a large number of points $(t_i,y_i)$, it is convenient to organize the implementation of Euler's method in a
table as shown.  We begin with the given initial data.

\begin{center}
\begin{tabular}{|c|c|c|c|c|}
  \hline
  $t_i$&$y_i$&$dy/dt$&$\Delta y$\\
  \hline
  0.0000&1.0000&\hphantom{-1.0000} & \hphantom{-0.2000}\\
  \hline
\end{tabular}
\end{center}
\noindent
From here, we compute the slope of the tangent line $m=dy/dt$ using
the formula for $dy/dt$ from the differential 
equation, and then we find $\Delta y$, the change in $y$, using the rule $\Delta y = m\Delta t$.

\begin{center}
\begin{tabular}{|c|c|c|c|c|}
  \hline
  $t_i$&$y_i$&$dy/dt$&$\Delta y$\\
  \hline
  0.0000&1.0000&-1.0000&-0.2000\\
  \hline
\end{tabular}
\end{center}
\noindent
Next, we increase $t_i$ by $\Delta t$ and $y_i$ by $\Delta y$ to get
\noindent
\medskip
\begin{center}
\begin{tabular}{|c|c|c|c|c|}
  \hline
  $t_i$&$y_i$&$dy/dt$&$\Delta y$\\
  \hline
  0.0000&1.0000&-1.0000&-0.2000\\ \hline
  0.2000&0.8000& & \\
  \hline
\end{tabular}
\end{center}
\noindent
and then we simply continue the process for however many steps we decide, eventually generating a table like the one that follows.
\medskip
\begin{center}
\begin{tabular}{|c|c|c|c|c|}
  \hline
  $t_i$&$y_i$&$dy/dt$&$\Delta y$\\
  \hline
  0.0000&1.0000&-1.0000&-0.2000\\ \hline
  0.2000&0.8000&-0.6000&-0.1200\\ \hline
  0.4000&0.6800&-0.2800&-0.0560\\ \hline
  0.6000&0.6240&-0.0240&-0.0048\\ \hline
  0.8000&0.6192&0.1808&0.0362\\ \hline
  1.0000&0.6554&0.3446&0.0689\\ \hline
  1.2000&0.7243&0.4757&0.0951\\ 
  \hline
\end{tabular}
\end{center}



\input{activities/7.3.Act1}

\newpage

\input{activities/7.3.Act2}

\subsection*{The error in Euler's method} \index{Euler's Method!error}

Since we are approximating the solutions to an initial value problem
using tangent lines, we should expect that the error in the
approximation will be less when the step size is smaller.
To explore this observation quantitatively, let's consider the initial value problem
$$
  \frac{dy}{dt} = y, \ y(0) = 1
$$
whose solution we can easily find.

Consider the question posed by this initial value problem: ``what function do we know that is the same as its own
derivative and has value 1 when $t=0$?'' It is not hard to see that the 
solution is $y(t) = e^t$.    We now apply Euler's method
to approximate $y(1) = e$ using several values of $\Delta t$.  These
approximations will be denoted by $E_{\Delta t}$, and these estimates provide us a way to see how accurate Euler's Method is.

To begin, we apply Euler's method with a step size of $\Delta t =
0.2$.  In that case, we find that $y(1) \approx E_{0.2} = 2.4883$.  The
error is therefore $y(1) - E_{0.2} = e - 2.4883 \approx 0.2300$.

Repeatedly halving $\Delta t$ gives the following
results, expressed in both tabular and graphical form.

\begin{center}
\begin{minipage}[c]{2.0in}
\begin{tabular}{|c|c|c|}
  \hline
  $\Delta t$ & $E_{\Delta t}$ & Error \\
  \hline
  0.200&2.4883&0.2300\\
  0.100&2.5937&0.1245\\
  0.050&2.6533&0.0650\\
  0.025&2.6851&0.0332\\
  \hline
\end{tabular}
\end{minipage}
\hspace*{0.25in}
\begin{minipage}[c]{2.0in}
\includegraphics{figures/7_3_error.eps}
\end{minipage}
\end{center}

Notice, both numerically and graphically, that the error is roughly halved when $\Delta t$ is halved.
This example illustrates the following general principle.

\vspace*{5pt}
\nin \framebox{\hspace*{3 pt}
\parbox{\boxwidth}{
    If Euler's method is to approximate the solution to an initial
    value problem at a point $\overline{t}$, then the error is
    proportional to $\Delta t$.  That is,
    $$
    y(\overline{t}) - E_{\Delta t} \approx K\Delta t
    $$
    for some constant of proportionality $K$.

} \hspace*{3 pt}}
\vspace*{1pt}

\newpage

\begin{summary}
\item Euler's method is an algorithm for approximating the solution to
  an initial value problem by following the tangent lines while we
  take horizontal steps across the $t$-axis.
\item If we wish to approximate $y(\overline{t})$ for some fixed
  $\overline{t}$ by taking horizontal steps of size $\Delta t$, then the
  error in our approximation is proportional to $\Delta t$.  
\end{summary}

\nin \hrulefill

\input{exercises/7.3} 



\clearpage
