\begin{exercises} 

\item A certain function $y=p(x)$ has its local linearization at $a = 3$ given by $L(x) = -2x + 5$.

\ba
	\item What are the values of $p(3)$ and $p'(3)$?  Why?
	\item Estimate the value of $p(2.79)$.
	\item Suppose that $p''(3) = 0$ and you know that $p''(x) < 0$ for $x < 3$.  Is your estimate in (b) too large or too small?
	\item Suppose that $p''(x) > 0$ for $x > 3$.  Use this fact and the additional information above to sketch an accurate graph of $y = p(x)$ near $x = 3$.  Include a sketch of $y = L(x)$ in your work.
\ea

\item A potato is placed in an oven, and the potato's temperature $F$ (in degrees Fahrenheit) at various points in time is taken and recorded in the following table. Time $t$ is measured in minutes.

\begin{tabular}{| l || l |}
\hline
$t$ & $F(t)$ \\ \hline \hline
0 & 70\\ \hline
15 & 180.5 \\ \hline
30 & 251 \\ \hline
45 & 296 \\ \hline
60 & 324.5 \\ \hline
75 & 342.8 \\ \hline
90 & 354.5  \\ \hline
\end{tabular}

\ba
	\item Use a central difference to estimate $F'(60)$.  Use this estimate as needed in subsequent questions.
	\item Find the local linearization $y = L(t)$ to the function $y = F(t)$ at the point where $a = 60$.
	\item Determine an estimate for $F(63)$ by employing the local linearization.  
	\item Do you think your estimate in (c) is too large or too small?  Why?
\ea

\item An object moving along a straight line path has a differentiable position function $y = s(t)$.  It is known that at time $t = 9$ seconds, the object's position is $s = 4$ feet (measured from its starting point at $t = 0$).  Furthermore, the object's instantaneous velocity at $t = 9$ is $-1.2$ feet per second, and its acceleration at the same instant is $0.08$ feet per second per second.

\ba
	\item Use local linearity to estimate the position of the object at $t = 9.34$.
	\item Is your estimate likely too large or too small?  Why?
	\item In everyday language, describe the behavior of the moving object at $t = 9$.  Is it moving toward its starting point or away from it? Is its velocity increasing or decreasing?
\ea

\item For a certain function $f$, its derivative is known to be $f'(x) = (x-1)e^{-x^2}$.  Note that you do not know a formula for $y = f(x)$.
\ba  
  	\item At what $x$-value(s) is $f'(x) = 0$?  Justify your answer algebraically, but include a graph of $f'$ to support your conclusion.
	\item Reasoning graphically, for what intervals of $x$-values is $f''(x) > 0$?  What does this tell you about the  behavior of the original function $f$?  Explain.
	\item Assuming that $f(2) = -3$, estimate the value of $f(1.88)$ by finding and using the tangent line approximation to $f$ at $x=2$.  Is your estimate larger or smaller than the true value of $f(1.88)$?  Justify your answer.
\ea



\end{exercises}
\afterexercises