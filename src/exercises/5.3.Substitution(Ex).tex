\begin{exercises} 
  \item This problem centers on finding antiderivatives for the basic trigonometric functions other than $\sin(x)$ and $\cos(x)$.
  \ba
  	\item Consider the indefinite integral $\ds \int \tan(x) \, dx$.  By rewriting the integrand as $\tan(x) = \frac{\sin(x)}{\cos(x)}$ and identifying an appropriate function-derivative pair, make a $u$-substitution and hence evaluate $\ds \int \tan(x) \, dx$.
	\item In a similar way, evaluate $\ds \int \cot(x) \, dx$.
	\item Consider the indefinite integral 
	$$\int \frac{\sec^2(x) + \sec(x) \tan(x)}{\sec(x) + \tan(x)} \, dx.$$
	Evaluate this integral using the substitution $u = \sec(x) + \tan(x)$.
	\item Simplify the integrand in (c) by factoring the numerator.  What is a far simpler way to write the integrand?
	\item Combine your work in (c) and (d) to determine $\int \sec(x) \, dx$.
	\item Using (c)-(e) as a guide, evaluate $\ds \int \csc(x) \, dx$.
  \ea

  \item Consider the indefinite integral $\ds \int x \sqrt{x-1} \, dx.$
  \ba
  	\item At first glance, this integrand may not seem suited to substitution due to the presence of $x$ in separate locations in the integrand.  Nonetheless, using the composite function $\sqrt{x-1}$ as a guide, let $u = x-1$.  Determine expressions for both $x$ and $dx$ in terms of $u$.
	\item Convert the given integral in $x$ to a new integral in $u$.
	\item Evaluate the integral in (b) by noting that $\sqrt{u} = u^{1/2}$ and observing that it is now possible to rewrite the integrand in $u$ by expanding through multiplication.
	\item Evaluate each of the integrals $\ds \int x^2 \sqrt{x-1} \, dx$ and $\ds \int x \sqrt{x^2 - 1} \, dx$.  Write a paragraph to discuss the similarities among the three indefinite integrals in this problem and the role of substitution and algebraic rearrangement in each.
  \ea

  \item Consider the indefinite integral $\ds \int \sin^3(x) \, dx$. 
  \ba
  	\item Explain why the substitution $u = \sin(x)$ will not work to help evaluate the given integral.
	\item Recall the Fundamental Trigonometric Identity, which states that $\sin^2(x) + \cos^2(x) = 1$.  By observing that $\sin^3(x) = \sin(x) \cdot \sin^2(x)$, use the Fundamental Trigonometric Identity to rewrite the integrand as the product of $\sin(x)$ with another function.
	\item Explain why the substitution $u = \cos(x)$ now provides a possible way to evaluate the integral in (b).
	\item Use your work in (a)-(c) to evaluate the  indefinite integral $\ds \int \sin^3(x) \, dx$.
	\item Use a similar approach to evaluate $\ds \int \cos^3(x) \, dx$.
  \ea
  
  \item For the town of Mathland, MI, residential power consumption has shown certain trends over recent years.  Based on data reflecting average usage, engineers at the power company have modeled the town's rate of energy consumption by the function
 $$r(t) = 4 + \sin(0.263t + 4.7) + \cos(0.526t+9.4).$$
Here, $t$ measures time in hours after midnight on a typical weekday, and $r$ is the rate of consumption in megawatts\footnote{The unit \emph{megawatt} is itself a rate, which measures energy consumption per unit time.  A \emph{megawatt-hour} is the total amount of energy that is equivalent to a constant stream of 1 megawatt of power being sustained for 1 hour.} at time $t$. 
Units are critical throughout this problem.
	\ba
		\item Sketch a carefully labeled graph of $r(t)$ on the interval [0,24] and explain its meaning.  Why is this a reasonable model of power consumption?
		\item Without calculating its value, explain the meaning of $\int_0^{24} r(t) \, dt$.   Include appropriate units on your answer.
		
  		\item Determine the exact amount of power Mathland consumes in a typical day.  
		\item What is Mathland's average rate of energy consumption in a given 24-hour period?  What are the units on this quantity?
	\ea
\end{exercises}
\afterexercises
