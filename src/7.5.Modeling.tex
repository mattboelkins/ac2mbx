\section{Modeling with differential equations} \label{S:7.5.Modeling}

\vspace*{-14 pt}
\framebox{\hspace*{3 pt}
\parbox{\boxwidth}{\begin{goals}
\item How can we use differential equations to describe 
  phenomena in the world around us?
\item How can we use differential equations to better understand these
  phenomena? 
\end{goals}} \hspace*{3 pt}}

\subsection*{Introduction}

In our work to date, we have seen several ways that
differential equations arise in the natural world, from the growth of
a population to the temperature of a cup of coffee.  In this section,
we will look more closely at how differential equations give us a
natural way to describe various phenoma.  As we'll see, the key is to
focus on understanding the different factors that cause a quantity to
change.

\input{previews/7.5}

\subsection*{Developing a differential equation}

Preview activity~\ref{PA:7.5} demonstrates the kind of thinking we will be
doing in this section.  In each of the two examples we considered, there is a
quantity, such as the amount of money in the bank account or the
amount of salt in the tank, that is changing due to several factors.
The governing differential equation results from the total rate of change being the difference between the rate of
increase and the rate of decrease.

\bex \label{Ex:7.5.1}
In the Great Lakes region, rivers flowing into the lakes carry a great
deal of pollution in the form of small pieces of plastic averaging 1
millimeter in diameter.  In order to understand how the amount of
plastic in Lake Michigan is changing, construct a model for how this type pollution has built up in the lake.
\eex

First, some basic facts about Lake Michigan.
\begin{itemize}
  \item The volume of the lake is
    $5\cdot10^{12}$ cubic meters.
  \item Water flows into the lake at a rate of
    $5\cdot10^{10}$ cubic meters per year.  It flows out of the lake
    at the same rate.
  \item Each cubic meter flowing
    into the lake contains roughly $3\cdot10^{-8}$ cubic meters of
    plastic pollution.
\end{itemize}

Let's denote the amount of pollution in the lake by $P(t)$, where $P$
is measured in cubic meters of plastic and $t$ in years.  Our goal
is to describe the rate of change of this function;  in other
words, we want to develop a differential equation describing $P(t)$.

First, we will measure how $P(t)$ increases due to pollution flowing
into the lake.  We know that $5\cdot10^{10}$ cubic meters of water
enters the lake every year and each cubic meter of water contains
$3\cdot10^{-8}$ cubic meters of pollution.  Therefore, pollution
enters the lake at the rate of
$$
\left(5\cdot 10^{10} \frac{m^3 \mbox{\ water}}{\mbox{year}}\right) \cdot \left(3\cdot10^{-8} \frac{m^3 \mbox{\ plastic}}{m^3 \mbox{\ water}} \right) = 1.5\cdot 10^3\quad
\hbox{cubic m of plastic per year}.
$$

Second, we will measure how $P(t)$ decreases due to pollution flowing
out of the lake.  If the total amount of pollution is $P$ cubic
meters and the volume of Lake Michigan is $5\cdot 10^{12}$ cubic
meters, then the concentration of plastic pollution in Lake Michigan is
$$
\frac{P}{5\cdot10^{12}} \quad \hbox{cubic meters of plastic per cubic meter of water}.
$$
Since $5\cdot10^{10}$ cubic meters of water flow out each year\footnote{and we assume that each cubic meter of water that flows out carries with it the plastic pollution it contains}, then
the plastic pollution leaves the lake at the rate of
$$
\left(\frac{P}{5\cdot10^{12}} \frac{m^3 \mbox{\ plastic}}{m^3 \mbox{\ water}} \right) \cdot \left(5\cdot10^{10} \frac{m^3 \mbox{\ water}}{\mbox{year}} \right)=\frac{P}{100} 
\quad \hbox{cubic meters of plastic
  per year}.
$$

The total rate of change of $P$ is thus the difference between the rate at which
pollution enters the lake minus the rate at which pollution leaves the
lake;  that is,
\begin{eqnarray*}
\frac{dP}{dt} & = &1.5\cdot10^{3}-\frac{P}{100} \\
                   & = & \frac{1}{100}(1.5\cdot10^{5} - P).
\end{eqnarray*}

We have now found a differential equation that describes the rate
at which the amount of pollution is changing.  To better understand the
behavior of $P(t)$, we now apply some
of the techniques we have recently developed.

Since this is an autonomous differential equation, we can sketch
$dP/dt$ as a function of $P$ and then construct a slope field, as shown in Figure~\ref{F:7.5.Ex1}.

\begin{figure}[h]
\begin{center}
  \includegraphics{figures/7_5_lake_michigan.eps}\qquad
  \includegraphics{figures/7_5_slope_field.eps}
\caption{Plots of $\frac{dP}{dt}$ vs. $P$ and the slope field for the differential equation $\frac{dP}{dt} = \frac{1}{100}(1.5\cdot10^{5} - P)$.} \label{F:7.5.Ex1}
\end{center}
\end{figure}

These plots both show that $P=1.5\cdot10^5$ is a stable equilibrium.  Therefore,
we should expect that the amount of pollution in Lake Michigan will
stabilize near $1.5\cdot10^5$ cubic meters of pollution.

Next, assuming that there is initially no pollution in the lake, we will
solve the initial value problem
$$
\frac{dP}{dt} = \frac{1}{100}(1.5\cdot10^{5} - P), \ P(0) = 0.
$$
Separating variables, we find that
$$
\frac1{1.5\cdot10^5-P} \frac{dP}{dt} = \frac1{100}.
$$
Integrating with respect to $t$, we have 
$$  \int \frac1{1.5\cdot10^5-P} \frac{dP}{dt}~dt = \int \frac1{100}~dt,$$
and thus changing variables on the left and antidifferentiating on both sides, we find that
\begin{eqnarray*}
  \int \frac{dP}{1.5\cdot10^5-P} &=& \int \frac1{100}~dt \\
  -\ln|1.5\cdot10^5 - P| & = & \frac1{100}t + C
\end{eqnarray*}
Finally, multiplying both sides by $-1$ and using the definition of the logarithm, we find that
\begin{equation} \label{E:7.5.Ex1C}  1.5\cdot10^5 - P = C e^{-t/100}.
\end{equation}
This is a good time to determine the constant $C$.  Since $P =
0$ when $t=0$, we have
$$
1.5\cdot 10^5 - 0 = Ce^0 = C.
$$
In other words, $C=1.5\cdot10^5$. 

Using this value of $C$ in Equation~(\ref{E:7.5.Ex1C}) and solving for $P$, we arrive at the solution
$$ P(t) = 1.5\cdot10^5(1-e^{-t/100}).$$
Superimposing the graph of $P$ on the slope field we saw in Figure~\ref{F:7.5.Ex1}, we see, as shown in Figure~\ref{F:7.5.Ex1a}
\begin{figure}[h]
\begin{center}
  \includegraphics{figures/7_5_solution.eps}
\caption{The solution $P(t)$ and the slope field for the differential equation $\frac{dP}{dt} = \frac{1}{100}(1.5\cdot10^{5} - P)$.} \label{F:7.5.Ex1a}
\end{center}
\end{figure}
We see that, as expected, the amount of plastic pollution stabilizes around
$1.5\cdot10^5$ cubic meters.

\afterex

There are many important lessons to learn from Example~\ref{Ex:7.5.1}.  Foremost is how we can develop a differential equation by thinking about the ``total rate = rate in - rate out'' model.  In addition, we note how we can bring together all of our available understanding (plotting $\frac{dP}{dt}$ vs. $P$, creating a slope field, solving the differential equation) to see how the differential equation describes the behavior of a changing quantity.

Of course, we can also explore what happens when certain aspects of the problem change.  For instance, let's suppose we are at a time when the plastic pollution entering Lake Michigan has
stabilized at $1.5\cdot10^5$ cubic meters, and that new legislation is
passed to prevent this type of pollution entering the lake.  So, there is no longer any inflow of plastic pollution to the lake.  How does the amount of plastic pollution in Lake Michigan now change?  For example, how long does it take for the amount of plastic pollution in the lake to halve?

Restarting the problem at time $t=0$, we now have the modified initial value problem
$$
\frac{dP}{dt} = -\frac{1}{100}P, \ P(0) = 1.5\cdot10^5.
$$
It is a straightforward and familiar exercise to find that the solution to this equation is $P(t) = 1.5\cdot10^5
e^{-t/100}$.  The time that it takes for half of the pollution to flow
out of the lake is given by $T$ where $P(T) = 0.75\cdot10^5$.  Thus, we must solve the equation
$$0.75\cdot10^5 = 1.5\cdot10^5e^{-T/100},$$
or
$$ \frac12 = e^{-T/100}.$$
It follows that 
$$T = -100\thinspace\ln\left(\frac12\right) \approx 69.3 \quad\hbox{years.}$$

In the upcoming activities, we explore some other natural settings in which differential equation model changing quantities.

\newpage

\input{activities/7.5.finance}

\input{activities/7.5.iv.drug}

\begin{summary}
\item Differential equations arise in a situation when we understand
  how various factors cause a quantity to change.
\item We may use the tools we have developed so far---slope
  fields, Euler's methods, and our method for solving separable
  equations---to understand a quantity described by a differential
  equation. 
\end{summary}

\nin \hrulefill

\input{exercises/7.5} 




\clearpage
