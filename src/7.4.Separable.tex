\section{Separable differential equations} \label{S:7.4.Separable}

\vspace*{-14 pt}
\framebox{\hspace*{3 pt}
\parbox{\boxwidth}{\begin{goals}
\item What is a separable differential equation?  
\item How can we find solutions to a separable differential equation?
\item Are some of the differential equations that arise in applications separable?
\end{goals}} \hspace*{3 pt}}


\subsection*{Introduction}

In Sections~\ref{S:7.2.Qualitative} and~\ref{S:7.3.Euler}, we have seen several ways to approximate the solution to
an initial value problem.   Given the frequency with which differential equations arise in the world around us, we would like to have some techniques
for finding explicit algebraic solutions of certain initial value
problems.  In this section, we focus on a particular class of differential
equations (called {\em separable}) and develop a method for finding
algebraic formulas for solutions to these equations.

A {\em separable differential equation}\index{separable} is a differential equation whose algebraic structure permits the variables present to be separated in a particular way.  For instance, consider the equation
$$
\frac{dy}{dt} = ty.
$$
We would like to separate the variables $t$ and $y$ so that all
occurrences of $t$ appear on the right-hand side, and all occurrences
of $y$ appears on the left and multiply $dy/dt$.  We may do this in the preceding differential equation by dividing both sides by
$y$:
$$
\frac1y\frac{dy}{dt} = t.
$$
Note particularly that when we attempt to separate the variables in a differential equation, we require that the left-hand side be a product in which the derivative $dy/dt$ is one term.  

Not every differential equation is separable.  For example, if we consider the
equation
$$
\frac{dy}{dt} = t-y,
$$
it may seem natural to separate it by writing
$$
y + \frac{dy}{dt} = t.
$$
As we will see, this will not be helpful since the left-hand side is
not a product of a function of $y$ with $\frac{dy}{dt}$.

\begin{pa} \label{PA:7.4}  In this preview activity, we explore whether certain differential equations are separable or not, and then revisit some key ideas from earlier work in integral calculus.
  \ba
\item Which of the following differential equations are separable?  If the equation is separable, write the equation in the revised form $g(y) \frac{dy}{dt} = h(t)$.
  \begin{enumerate}
    \item $\displaystyle \frac{dy}{dt} = -3y$.
    \item $\displaystyle \frac{dy}{dt} = ty - y$.
    \item $\displaystyle \frac{dy}{dt} = t + 1$.
    \item $\displaystyle \frac{dy}{dt} = t^2 - y^2$.
  \end{enumerate}
\item Explain why any autonomous differential equation is guaranteed to be separable.  
\item Why do we include the term ``$+C$'' in the expression $$\int x~dx =
  \frac{x^2}{2} + C?$$  
\item Suppose we know that a certain function $f$ satisfies the equation
  $$
  \int f'(x)~dx = \int x~dx.
  $$
  What can you conclude about $f$?
\ea
\end{pa} 
\afterpa


\subsection*{Solving separable differential equations} \index{separable}

Before we discuss a general approach to solving a separable differential equation, it is instructive to consider an example.
\bex \label{Ex:7.4.1}
Find all functions $y$ that are solutions to the differential equation 
$$\frac{dy}{dt}= \frac{t}{y^2}.$$
\eex
We begin by separating the variables and writing
    $$
    y^2\frac{dy}{dt} = t.
    $$
Integrating both sides of the equation with respect to the independent
    variable $t$ shows that
    $$
    \int y^2\frac{dy}{dt}~dt = \int t~dt.
    $$
Next, we notice that the left-hand side allows us to change
    the variable of antidifferentiation\footnote{This is why we required that the left-hand side be written as a
    product in which $dy/dt$ is one of the terms.} from $t$ to $y$.  In
    particular,
    $dy = \frac{dy}{dt}~dt$, so we now have
    $$
    \int y^2 ~dy = \int t~dt.
    $$
   This most recent equation says that two families of antiderivatives are
    equal to one another.  Therefore, when we find representative
    antiderivatives of both sides, we know they must differ by
    arbitrary constant $C$.  Antidifferentiating and including the integration constant $C$ on the right, we find that
    $$
    \frac{y^3}{3} = \frac{t^2}{2} + C.
    $$
    Again, note that it is not necessary to include an arbitrary constant on both sides 
    of the equation;  we know that $y^3/3$ and $t^2/2$ are in the same
    family of antiderivatives and must therefore differ by a single
    constant.

Finally, we may now solve the last equation above for $y$ as a function of $t$, which gives
    $$
    y(t) = \sqrt[3]{\frac 32 \thinspace t^2 + 3C}.
    $$
    Of course, the term $3C$ on the right-hand side represents
    3 times an unknown constant.  It is, therefore, still an unknown
    constant, which we will rewrite as $C$.  We thus conclude that the funtion
    $$
    y(t) = \sqrt[3]{\frac 32 \thinspace t^2 + C}
    $$
is a solution to the original differential equation for any value of $C$.

Notice that because this solution depends on the arbitrary constant $C$, we have found an infinite family of
solutions.  This makes sense because we expect to find a unique solution that corresponds to any given
 initial value.

For example, if we want to solve the initial value problem
$$
  \frac{dy}{dt} = \frac{t}{y^2}, \
  y(0) = 2,
$$
we know that the solution has the form $y(t) = \sqrt[3]{\frac32\thinspace
  t^2 + C}$ for some constant $C$.  We therefore must find the appropriate
value for $C$ that gives the initial value $y(0)=2$.  Hence,
$$
  2 = y(0)  \sqrt[3]{\frac 32 \thinspace 0^2 + C} = \sqrt[3]{C},
  $$
which shows that $C = 2^3 = 8$.  The solution to the initial value problem is then
$$
y(t) = \sqrt[3]{\frac32\thinspace t^2+8}.
$$
\afterex

The strategy of Example~\ref{Ex:7.4.1} may be applied to any differential equation of the form $\frac{dy}{dt} = g(y) \cdot h(t)$, and any differential equation of this form is said to be \emph{separable}.  We work to solve a separable differential equation by writing
$$\frac{1}{g(y)} \frac{dy}{dt} = h(t),$$ 
and then integrating both sides with respect to $t$.  After integrating, we strive to solve algebraically for $y$ in order to write $y$ as a function of $t$.

We consider one more example before doing further exploration in some activities.

\bex \label{Ex:7.4.2}
Solve the differential equation
$$\frac{dy}{dt} =3y.$$
\eex
Following the same strategy as in Example~\ref{Ex:7.4.1}, we have
$$  \frac 1y \frac{dy}{dt} = 3. $$
Integrating both sides with respect to $t$,
$$  \int \frac 1y\frac{dy}{dt}~dt = \int 3~dt,$$
and thus 
$$ \int \frac 1y~dy =  \int 3~dt.$$
Antidifferentiating and including the integration constant, we find that
$$  \ln|y| = 3t + C.$$
Finally, we need to solve for $y$.  Here, one point deserves careful
attention.  By the definition of the natural logarithm function, it follows that
$$
|y| = e^{3t+C} = e^{3t}e^C.
$$
Since $C$ is an unknown constant, $e^C$ is as well, though we do know
that it is positive (because $e^x$ is positive for any $x$).
When we remove the absolute value in order to solve for $y$, however, this constant may be either positive or
negative.  We 
will denote this updated constant (that accounts for a possible $+$ or $-$) by $C$ to obtain
$$
y(t) = Ce^{3t}.
$$

There is one more slightly technical point to make.  Notice that $y=0$
is an equilibrium solution to this differential equation.  In solving
the equation above, we begin by dividing both sides by $y$, which
is not allowed if $y=0$.  To be perfectly careful, therefore, we will typically
consider the equilibrium solutions separably.  In this case, notice that the final
form of our solution captures the equilibrium solution by allowing
$C=0$. 
\afterex

\begin{activity} \label{A:7.4.1}  
Suppose that the population of a town is growing continuously at an annual rate of 3\% per year.

\ba
\item Let $P(t)$ be the population of the town in year $t$.  Write a
  differential equation that describes the annual growth rate.

\item Find the solutions of this differential equation.

\item If you know that the town's population in year 0 is 10,000, find
  the population $P(t)$.

\item How long does it take for the population to double?  This time
  is called the {\em doubling time}.

\item Working more generally, find the doubling time if the annual
  growth rate is $k$ times the population.

\ea
\end{activity}
\begin{smallhint}
\ba
	\item Small hints for each of the prompts above.
\ea
\end{smallhint}
\begin{bighint}
\ba
	\item Big hints for each of the prompts above.
\ea
\end{bighint}
\begin{activitySolution}
\ba
	\item Solutions for each of the prompts above.
\ea
\end{activitySolution}
\aftera

\begin{activity} \label{A:7.4.2}  
  Suppose that a cup of coffee is initially at a temperature of
  105$^\circ$ F and is placed in a 75$^\circ$ F room.  Newton's law of
  cooling says that 
  $$
  \frac{dT}{dt} = -k(T-75),
  $$ 
  where $k$ is a constant of proportionality.

\ba
\item Suppose you measure that the coffee is cooling at one degree per
  minute at the time the coffee is brought into the room.  Use the
  differential equation to determine the value of the constant $k$.

\item Find all the solutions of this differential equation.

\item What happens to all the solutions as $t\to\infty$?  Explain how
  this agrees with your intuition.

\item What is the temperature of the cup of coffee after 20 minutes?

\item How long does it take for the coffee to cool to 80$^\circ$?

\ea
\end{activity}
\begin{smallhint}
\ba
	\item Small hints for each of the prompts above.
\ea
\end{smallhint}
\begin{bighint}
\ba
	\item Big hints for each of the prompts above.
\ea
\end{bighint}
\begin{activitySolution}
\ba
	\item Solutions for each of the prompts above.
\ea
\end{activitySolution}
\aftera

\begin{activity} \label{A:7.4.3}  Solve each of the following differential equations or initial value problems.
\ba
	\item $\ds \frac{dy}{dt} - (2-t) y = 2-t$
	\item $\ds \frac{1}{t}\frac{dy}{dt} = e^{t^2-2y}$
	\item $y' = 2y+2$, \ \ $y(0)=2$
	\item $y' = 2y^2$, \ \ $y(-1) = 2$
	\item $\ds \frac{dy}{dt} = \frac{-2ty}{t^2 + 1}$, \ \  $y(0) = 4$
\ea
\end{activity}
\begin{smallhint}
\ba
	\item Small hints for each of the prompts above.
\ea
\end{smallhint}
\begin{bighint}
\ba
	\item Big hints for each of the prompts above.
\ea
\end{bighint}
\begin{activitySolution}
\ba
	\item Solutions for each of the prompts above.
\ea
\end{activitySolution}
\aftera


\begin{summary}
\item A separable differential equation is one that may be rewritten
  with all occurrences of the dependent variable multiplying the
  derivative and all occurrences of the independent variable on the
  other side of the equation.
\item We may find the solutions to certain separable differential equations
  by separating variables, integrating with respect to $t$, and ultimately solving the resulting algebraic equation for $y$.
 
\item This technique allows us to solve many important differential
  equations that arise in the world around us.  For instance, questions of
  growth and decay and Newton's Law of Cooling give rise to separable
  differential equations.  Later, we will learn in Section~\ref{S:7.6.Logistic} that the important logistic differential equation is also separable.
\end{summary}

\nin \hrulefill

\newpage

\begin{exercises} 
  \item  The mass of a radioactive sample decays at a rate that is
    proportional to its mass. 
    \ba
  \item Express this fact as a differential equation for the mass
    $M(t)$ using $k$ for the constant of proportionality.
    \item If the initial mass is $M_0$, find an expression for the
      mass $M(t)$.
    \item The {\em half-life} of the sample is the amount of time
      required for half of the mass to decay.  Knowing that the
      half-life of Carbon-14 is 5730 years, find the value of $k$ for
      a sample of Carbon-14.
    \item How long does it take for a sample of Carbon-14 to be
      reduced to one-quarter its original mass?
    \item Carbon-14 naturally occurs in our environment; any
      living organism takes in Carbon-14 when it eats and breathes.  Upon
      dying, however, the organism no longer takes in Carbon-14. 
      
      Suppose that you find remnants of a pre-historic firepit.  By
      analyzing the charred wood in the pit, you determine that the
      amount of Carbon-14 is only 30\% of the amount in living trees.
      Estimate the age of the firepit.\footnote{This approach is the basic idea behind radiocarbon dating.}

      
    \ea

  \item Consider the initial value problem
    
    $$  \frac{dy}{dt} = -\frac ty, \ y(0) = 8$$
    

    \ba
    \item Find the solution of the initial value problem and sketch its
      graph.

    \item For what values of $t$ is the solution defined?  

    \item What is the value of $y$ at the last time that the
      solution is defined? 

    \item By looking at the differential equation, explain why we
      should not expect to find solutions with the value of $y$ you noted in (c).

      \ea

  \item  Suppose that a cylindrical water tank with a hole in the
    bottom is filled with water.  The water, of course, will leak out
    and the height of the water will decrease.  Let $h(t)$ denote the
    height of the water.  A physical principle called {\em Torricelli's
      Law} implies that the height decreases at a rate proportional to
    the square root of the height.

    \ba
    \item Express this fact using $k$ as the constant of
      proportionality.  
    \item Suppose you have two tanks, one with $k=1$ and another with
      $k=10$.  What physical differences would you expect to find?
    \item Suppose you have a tank for which the height decreases at 20
      inches per minute when the water is filled to a depth of 100
      inches.  Find the value of $k$.  
    \item Solve the initial value problem for the tank in part (c), and graph the solution you determine.
    \item How long does it take for the water to run out of the tank?
    \item Is the solution that you found valid for all time $t$?  If
      so, explain how you know this.  If not, explain why not.
    \ea

  \item The {\em Gompertz equation} is a model that is used to
    describe the growth of certain populations.  Suppose that $P(t)$
    is the population of some organism and that
    $$
    \frac{dP}{dt} = -P\ln\left(\frac P3\right) = -P(\ln P - \ln 3).
    $$

    \ba
    \item Sketch a slope field for $P(t)$ over the range $0\leq P\leq
      6$.

    \item Identify any equilibrium solutions and determine whether
      they are stable or unstable.

    \item Find the population $P(t)$ assuming that $P(0) = 1$ and sketch
      its graph.  What happens to $P(t)$ after a very long time?

    \item Find the population $P(t)$ assuming that $P(0) = 6$ and sketch
      its graph.  What happens to $P(t)$ after a very long time?

    \item Verify that the long-term behavior of your solutions agrees
      with what you predicted by looking at the slope field.

      \ea
        
\end{exercises}
\afterexercises
 



\clearpage
