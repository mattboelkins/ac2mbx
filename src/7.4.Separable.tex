\section{Separable differential equations} \label{S:7.4.Separable}

\vspace*{-14 pt}
\framebox{\hspace*{3 pt}
\parbox{\boxwidth}{\begin{goals}
\item What is a separable differential equation?  
\item How can we find solutions to a separable differential equation?
\item Are some of the differential equations that arise in applications separable?
\end{goals}} \hspace*{3 pt}}


\subsection*{Introduction}

In Sections~\ref{S:7.2.Qualitative} and~\ref{S:7.3.Euler}, we have seen several ways to approximate the solution to
an initial value problem.   Given the frequency with which differential equations arise in the world around us, we would like to have some techniques
for finding explicit algebraic solutions of certain initial value
problems.  In this section, we focus on a particular class of differential
equations (called {\em separable}) and develop a method for finding
algebraic formulas for solutions to these equations.

A {\em separable differential equation}\index{separable} is a differential equation whose algebraic structure permits the variables present to be separated in a particular way.  For instance, consider the equation
$$
\frac{dy}{dt} = ty.
$$
We would like to separate the variables $t$ and $y$ so that all
occurrences of $t$ appear on the right-hand side, and all occurrences
of $y$ appears on the left and multiply $dy/dt$.  We may do this in the preceding differential equation by dividing both sides by
$y$:
$$
\frac1y\frac{dy}{dt} = t.
$$
Note particularly that when we attempt to separate the variables in a differential equation, we require that the left-hand side be a product in which the derivative $dy/dt$ is one term.  

Not every differential equation is separable.  For example, if we consider the
equation
$$
\frac{dy}{dt} = t-y,
$$
it may seem natural to separate it by writing
$$
y + \frac{dy}{dt} = t.
$$
As we will see, this will not be helpful since the left-hand side is
not a product of a function of $y$ with $\frac{dy}{dt}$.

\input{previews/7.4.PA1}

\subsection*{Solving separable differential equations} \index{separable}

Before we discuss a general approach to solving a separable differential equation, it is instructive to consider an example.
\bex \label{Ex:7.4.1}
Find all functions $y$ that are solutions to the differential equation 
$$\frac{dy}{dt}= \frac{t}{y^2}.$$
\eex
We begin by separating the variables and writing
    $$
    y^2\frac{dy}{dt} = t.
    $$
Integrating both sides of the equation with respect to the independent
    variable $t$ shows that
    $$
    \int y^2\frac{dy}{dt}~dt = \int t~dt.
    $$
Next, we notice that the left-hand side allows us to change
    the variable of antidifferentiation\footnote{This is why we required that the left-hand side be written as a
    product in which $dy/dt$ is one of the terms.} from $t$ to $y$.  In
    particular,
    $dy = \frac{dy}{dt}~dt$, so we now have
    $$
    \int y^2 ~dy = \int t~dt.
    $$
   This most recent equation says that two families of antiderivatives are
    equal to one another.  Therefore, when we find representative
    antiderivatives of both sides, we know they must differ by
    arbitrary constant $C$.  Antidifferentiating and including the integration constant $C$ on the right, we find that
    $$
    \frac{y^3}{3} = \frac{t^2}{2} + C.
    $$
    Again, note that it is not necessary to include an arbitrary constant on both sides 
    of the equation;  we know that $y^3/3$ and $t^2/2$ are in the same
    family of antiderivatives and must therefore differ by a single
    constant.

Finally, we may now solve the last equation above for $y$ as a function of $t$, which gives
    $$
    y(t) = \sqrt[3]{\frac 32 \thinspace t^2 + 3C}.
    $$
    Of course, the term $3C$ on the right-hand side represents
    3 times an unknown constant.  It is, therefore, still an unknown
    constant, which we will rewrite as $C$.  We thus conclude that the funtion
    $$
    y(t) = \sqrt[3]{\frac 32 \thinspace t^2 + C}
    $$
is a solution to the original differential equation for any value of $C$.

Notice that because this solution depends on the arbitrary constant $C$, we have found an infinite family of
solutions.  This makes sense because we expect to find a unique solution that corresponds to any given
 initial value.

For example, if we want to solve the initial value problem
$$
  \frac{dy}{dt} = \frac{t}{y^2}, \
  y(0) = 2,
$$
we know that the solution has the form $y(t) = \sqrt[3]{\frac32\thinspace
  t^2 + C}$ for some constant $C$.  We therefore must find the appropriate
value for $C$ that gives the initial value $y(0)=2$.  Hence,
$$
  2 = y(0)  \sqrt[3]{\frac 32 \thinspace 0^2 + C} = \sqrt[3]{C},
  $$
which shows that $C = 2^3 = 8$.  The solution to the initial value problem is then
$$
y(t) = \sqrt[3]{\frac32\thinspace t^2+8}.
$$
\afterex

The strategy of Example~\ref{Ex:7.4.1} may be applied to any differential equation of the form $\frac{dy}{dt} = g(y) \cdot h(t)$, and any differential equation of this form is said to be \emph{separable}.  We work to solve a separable differential equation by writing
$$\frac{1}{g(y)} \frac{dy}{dt} = h(t),$$ 
and then integrating both sides with respect to $t$.  After integrating, we strive to solve algebraically for $y$ in order to write $y$ as a function of $t$.

We consider one more example before doing further exploration in some activities.

\bex \label{Ex:7.4.2}
Solve the differential equation
$$\frac{dy}{dt} =3y.$$
\eex
Following the same strategy as in Example~\ref{Ex:7.4.1}, we have
$$  \frac 1y \frac{dy}{dt} = 3. $$
Integrating both sides with respect to $t$,
$$  \int \frac 1y\frac{dy}{dt}~dt = \int 3~dt,$$
and thus 
$$ \int \frac 1y~dy =  \int 3~dt.$$
Antidifferentiating and including the integration constant, we find that
$$  \ln|y| = 3t + C.$$
Finally, we need to solve for $y$.  Here, one point deserves careful
attention.  By the definition of the natural logarithm function, it follows that
$$
|y| = e^{3t+C} = e^{3t}e^C.
$$
Since $C$ is an unknown constant, $e^C$ is as well, though we do know
that it is positive (because $e^x$ is positive for any $x$).
When we remove the absolute value in order to solve for $y$, however, this constant may be either positive or
negative.  We 
will denote this updated constant (that accounts for a possible $+$ or $-$) by $C$ to obtain
$$
y(t) = Ce^{3t}.
$$

There is one more slightly technical point to make.  Notice that $y=0$
is an equilibrium solution to this differential equation.  In solving
the equation above, we begin by dividing both sides by $y$, which
is not allowed if $y=0$.  To be perfectly careful, therefore, we will typically
consider the equilibrium solutions separably.  In this case, notice that the final
form of our solution captures the equilibrium solution by allowing
$C=0$. 
\afterex

\input{activities/7.4.Act1}

\input{activities/7.4.Act2}

\input{activities/7.4.Act3}


\begin{summary}
\item A separable differential equation is one that may be rewritten
  with all occurrences of the dependent variable multiplying the
  derivative and all occurrences of the independent variable on the
  other side of the equation.
\item We may find the solutions to certain separable differential equations
  by separating variables, integrating with respect to $t$, and ultimately solving the resulting algebraic equation for $y$.
 
\item This technique allows us to solve many important differential
  equations that arise in the world around us.  For instance, questions of
  growth and decay and Newton's Law of Cooling give rise to separable
  differential equations.  Later, we will learn in Section~\ref{S:7.6.Logistic} that the important logistic differential equation is also separable.
\end{summary}

\nin \hrulefill

\newpage

\input{exercises/7.4} 



\clearpage
