\begin{pa} \label{PA:6.4}  A bucket is being lifted from the bottom of a 50-foot deep well; its weight (including the water), $B$, in pounds at a height $h$ feet above the water is given by the function $B(h)$.  When the bucket leaves the water, the bucket and water together weigh $B(0) = 20$ pounds, and when the bucket reaches the top of the well, $B(50) = 12$ pounds.  Assume that the bucket loses water at a constant rate (as a function of height, $h$) throughout its journey from the bottom to the top of the well.

\ba
	\item Find a formula for $B(h)$.
	\item Compute the value of the product $B(5) \triangle h$, where $\triangle h = 2$ feet. Include units on your answer.  Explain why this product represents the approximate work it took to move the bucket of water from $h = 5$ to $h = 7$.
	\item Is the value in (b) an over- or under-estimate of the actual amount of work it took to move the bucket from $h = 5$ to $h = 7$?  Why?
	\item Compute the value of the product $B(22) \triangle h$, where $\triangle h = 0.25$ feet.  Include units on your answer.  What is the meaning of the value you found?
	\item More generally, what does the quantity $W_{\mbox{\small{slice}}} = B(h) \triangle h$ measure for a given value of $h$ and a small positive value of $\triangle h$?
	\item Evaluate the definite integral $\int_0^{50} B(h) \, dh$.  What is the meaning of the value you find?  Why?
\ea
\end{pa} 
\afterpa