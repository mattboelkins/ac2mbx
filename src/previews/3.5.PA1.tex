\begin{pa} \label{PA:3.5}
A spherical balloon is being inflated at a constant rate of 20 cubic inches per second.  How fast is the radius of the balloon changing at the instant the balloon's diameter is 12 inches?  Is the radius changing more rapidly when $d = 12$ or when $d = 16$?  Why?
\ba
	\item Draw several spheres with different radii, and observe that as volume changes, the radius, diameter, and surface area of the balloon also change.  
	\item Recall that the volume of a sphere of radius $r$ is $V = \frac{4}{3} \pi r^3$.  Note well that in the setting of this problem, \emph{both} $V$ and $r$ are changing as time $t$ changes, and thus both $V$ and $r$ may be viewed as implicit functions of $t$, with respective derivatives $\frac{dV}{dt}$ and $\frac{dr}{dt}$.  
	
	Differentiate both sides of the equation $V = \frac{4}{3} \pi r^3$ with respect to $t$ (using the chain rule on the right) to find a formula for $\frac{dV}{dt}$ that depends on both $r$ and $\frac{dr}{dt}$.
	\item At this point in the problem, by differentiating we have ``related the rates'' of change of $V$ and $r$.  Recall that we are given in the problem that the balloon is being inflated at a constant \emph{rate} of 20 cubic inches per second.  Is this rate the value of $\frac{dr}{dt}$ or $\frac{dV}{dt}$?  Why?
	\item From part (c), we know the value of $\frac{dV}{dt}$ at every value of $t$.  Next, observe that when the diameter of the balloon is 12, we know the value of the radius.  In the equation $\frac{dV}{dt} = 4\pi r^2 \frac{dr}{dt}$, substitute these values for the relevant quantities and solve for the remaining unknown quantity, which is $\frac{dr}{dt}$.  How fast is the radius changing at the instant $d = 12$?
	\item How is the situation different when $d = 16$?  When is the radius changing more rapidly, when $d = 12$ or when $d = 16$?
\ea
\end{pa} 
\afterpa