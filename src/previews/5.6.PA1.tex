\begin{pa} \label{PA:5.6}
As we begin to investigate ways to approximate definite integrals, it will be insightful to compare results to integrals whose exact values we know.  To that end, the following sequence of questions centers on $\int_0^3 x^2 \, dx$.
\ba
	\item Use the applet at \href{http://gvsu.edu/s/a9}{\texttt{http://gvsu.edu/s/a9}} with the function $f(x) = x^2$ on the window of $x$ values from $0$ to $3$ to compute $L_3$, the left Riemann sum with three subintervals.
	\item Likewise, use the applet to compute $R_3$ and $M_3$, the right and middle Riemann sums with three subintervals, respectively.
	\item Use the Fundamental Theorem of Calculus to compute the exact value of $I = \int_0^3 x^2 \, dx$.
	\item We define the \emph{error} in an approximation of a definite integral to be the difference between the integral's exact value and the approximation's value.  What is the error that results from using $L_3$? From $R_3$?  From $M_3$?
	\item In what follows in this section, we will learn a new approach to estimating the value of a definite integral known as the Trapezoid Rule.  The basic idea is to use trapezoids, rather than rectangles, to estimate the area under a curve.  What is the formula for the area of a trapezoid with bases of length $b_1$ and $b_2$ and height $h$?
	\item Working by hand, estimate the area under $f(x) = x^2$ on $[0,3]$ using three subintervals and three corresponding trapezoids.  What is the error in this approximation?  How does it compare to the errors you calculated in (d)?
\ea
\end{pa} 
\afterpa