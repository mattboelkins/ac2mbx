\begin{pa} \label{PA:2.6}
The equation $y = \frac{5}{9}(x-32)$ relates a temperature given in $x$ degrees Fahrenheit to the corresponding temperature $y$ measured in degrees Celcius.  
\ba
	\item Solve the equation $y = \frac{5}{9}(x-32)$ for $x$ to write $x$ (Fahrenheit temperature) in terms of $y$ (Celcius temperature).
	\item Let $C(x) = \frac{5}{9}(x-32)$ be the function that takes a Fahrenheit temperature as input and produces the Celcius temperature as output.  In addition,  let $F(y)$ be the function that converts a temperature given in $y$ degrees Celcius to the temperature $F(y)$ measured in degrees Fahrenheit.  Use your work in (a) to write a formula for $F(y)$.
	\item Next consider the new function defined by $p(x) = F(C(x))$.  Use the formulas for $F$ and $C$ to determine an expression for $p(x)$ and simplify this expression as much as possible.  What do you observe?
	\item Now, let $r(y) = C(F(y))$.  Use the formulas for $F$ and $C$ to determine an expression for $r(y)$ and simplify this expression as much as possible.  What do you observe?
	\item What is the value of $C'(x)$?  of $F'(y)$?  How do these values appear to be related?
\ea
\end{pa} 
\afterpa