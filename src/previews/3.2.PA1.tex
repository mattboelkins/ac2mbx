\begin{pa} \label{PA:3.2}
Let $a$, $h$, and $k$ be arbitrary real numbers with $a \ne 0$, and let  $f$ be the function given by the rule $f(x) = a(x-h)^2 + k$.
\ba
	\item What familiar type of function is $f$?  What information do you know about $f$ just by looking at its form? (Think about the roles of $a$, $h$, and $k$.)
	\item Next we use some calculus to develop familiar ideas from a different perspective.  To start, treat $a$, $h$, and $k$ as constants and compute $f'(x)$.
	\item Find all critical values of $f$. (These will depend on at least one of $a$, $h$, and $k$.)
	\item Assume that $a < 0$.  Construct a first derivative sign chart for $f$.
	\item Based on the information you've found above, classify the critical values of $f$ as maxima or minima.
\ea
\end{pa} \afterpa