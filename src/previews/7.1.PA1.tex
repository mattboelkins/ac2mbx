\begin{pa} \label{PA:7.1}
The position of a moving object is given by the
function $s(t)$, where $s$ is measured in feet and $t$ in seconds.  We
determine that the velocity is $v(t) = 4t + 1$ feet per second.
\ba
\item How much does the position change over the time interval
  $[0,4]$?
\item Does this give you enough information to determine $s(4)$, the
  position at time $t=4$?  If so, what is $s(4)$?  If not, what
  additional information would you need to know to determine $s(4)$?
\item Suppose you are told that the object's initial position $s(0) =
  7$.  Determine $s(2)$, the object's position 2 seconds later.
\item If you are told instead that the object's initial position is
  $s(0) = 3$, what is $s(2)$?
\item If we only know the velocity $v(t)=4t+1$, is it possible that the
  object's position at all times is $s(t) = 2t^2 + t - 4$?  Explain how
  you know.
\item Are there other possibilities for $s(t)$?  If so, what are they?  
\item If, in addition to knowing the velocity function is $v(t) = 4t+1$, we know the initial position $s(0)$, how many possibilities
  are there for $s(t)$?
\ea
\end{pa} 
\afterpa
